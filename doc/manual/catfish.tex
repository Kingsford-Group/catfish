\documentclass[letter,11pt]{article}
\usepackage[margin=1in]{geometry}
%\usepackage[body={6.5in,8.95in}, top=1.1in, left=0.9in, includefoot]{geometry}
%\usepackage{fourier}
\usepackage{pslatex}
\usepackage{graphicx}
\usepackage{caption}
\usepackage[obeyspaces]{url}
\usepackage{amsfonts}
\usepackage{amssymb}
%\usepackage{amsthm}
\usepackage{parskip}
\usepackage{mathrsfs}
\usepackage{multirow}
\usepackage[noblocks]{authblk}

\title{Catfish User Reference}

\author{Mingfu Shao\thanks{mingfu.shao@cs.cmu.edu}}
\author{Carl Kingsford\thanks{carlk@cs.cmu.edu}}
\affil{Computational Biology Department, School of Computer Science, Carnegie Mellon University}

\date{\today}

\begin{document}

\maketitle

\section{Installation}
To install Catfish, you need to first download Boost library, and then compile the source code of Catfish.

Download Boost from \url{http://www.boost.org}. Uncompress it
somewhere~(compiling and installing are not necessary). Set environment
variable \url{BOOST_HOME} to indicate the directory of Boost.
For example, for Unix platforms, add the following
statement to the file \url{~/.bash_profile}:\\
\makebox[0.9\textwidth][l]{\hspace{0.618cm}\url{export BOOST_HOME="/directory/to/your/boost/boost_1_60_0"} }

Get the source code of Catfish through \url{git}:\\
\makebox[0.9\textwidth][l]{\hspace{0.618cm}\url{$git clone git@github.com:shaomingfu/catfish.git .}}\\
Execute the following commands to generate \url{Makefile} and compile:\\
\makebox[0.9\textwidth][l]{\hspace{0.618cm}\url{$cd src} }\\
\makebox[0.9\textwidth][l]{\hspace{0.618cm}\url{$aclocal} }\\
\makebox[0.9\textwidth][l]{\hspace{0.618cm}\url{$autoconf} }\\
\makebox[0.9\textwidth][l]{\hspace{0.618cm}\url{$autoheader} }\\
\makebox[0.9\textwidth][l]{\hspace{0.618cm}\url{$automake -a} }\\
\makebox[0.9\textwidth][l]{\hspace{0.618cm}\url{$./configure } }\\
\makebox[0.9\textwidth][l]{\hspace{0.618cm}\url{$make} }\\
The executable file \url{catfish} will be present at \url{src/src}.
You might want to link it into \url{bin} through\\
\makebox[0.9\textwidth][l]{\hspace{0.618cm}\url{$cd bin} }\\
\makebox[0.9\textwidth][l]{\hspace{0.618cm}\url{$ln -sf ../src/src/catfish .} }

\section{Command line}
The usage of Catfish is as follows:\\
\makebox[0.9\textwidth][l]{\hspace{0.618cm}
	\url{$./catfish -i input.sgr/input.gtf -o output.out [-a algo]}}\\

\url{-i} parameter specifies the input file.
Catfish accepts two types of input file formats. The first one is \url{.sgr},
which specifies a directed acyclic graph. The first line of the file gives
$n$, indicating the number of vertices in the graph. The vertices 
are named from $0$ to $n - 1$, where vertex 0 has to be the source vertex
and vertex $n - 1$ has to be the sink vertex. 
Each of the following line specifies an edge, which consists of three integers:
the in-vertex, out-vertex and the weight of this edge. The second input file format
is \url{.gtf}. If it is this file format, Catfish will merge all transcripts
for each gene into a splice graph, and then try to decompose it.
There are two such input example files at \url{bin}.

\url{-o} parameter specifies the output file, which will show the predicted paths
and their associated abundances.

\url{-a} parameter specifies the algorithm.
There are three options: \url{full}, \url{core}, and \url{greedy}.
With option of \url{core}, the program will only run the core algorithm to partly
decompose the given splice graph, which will predict fewer paths but with
higher accuracy. With option of \url{full}, the program will completely
decompose the given splice graph, using greedy algorithm following the core part of the algorithm.
With option of \url{greedy}, the program will only use greedy algorithm to fully decompose
the given splice graph. This parameter is optional, and its default value is \url{full}.

\end{document}
